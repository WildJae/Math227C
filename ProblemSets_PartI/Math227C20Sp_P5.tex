\documentclass[12pt,letterpaper]{article}
\usepackage{amsmath,amsthm,amsfonts,amssymb,amscd}
\usepackage{fullpage}
\usepackage{lastpage}
\usepackage{enumerate}
\usepackage{fancyhdr}
\usepackage{mathrsfs}
\usepackage[margin=3cm,bottom=6cm]{geometry}
\usepackage{wrapfig}
\usepackage{graphicx}

\setlength{\parindent}{0.0in}
\setlength{\parskip}{0.05in}

\renewcommand{\theenumi}{\bf\Alph{enumi}}


% Edit these as appropriate
\newcommand\course{Math 227C}
\newcommand\semester{Spring 2019}     % <-- current semester
\newcommand\hwnum{3}                  % <-- homework number
\newcommand\yourname{Jun Allard} % <-- your name
%\newcommand\login{jcarberr}           % <-- your CS login

\newenvironment{answer}[1]{
  \subsubsection*{Problem \hwnum.#1}
}{\newpage}

\pagestyle{fancyplain}
\headheight 35pt
\lhead{ \course\ }
\chead{\textbf{ Problem Set 5}}
%\rhead{Due {\bf Friday, May 11th}}
\headsep 20pt

\begin{document}

The surface of a cell has a receptor where ligands become attached. 

\begin{enumerate}





%%%%%%%%%%%%%% PROBLEM %%%%%%%%%%%%%%%%%%

\item For antigen-detecting receptors like the T Cell Receptor, these ligands come in two types: agonists and antagonists. Assume that molecules arrive according to a Poisson process with rate $\lambda$ (in units of per-seconds). Among these molecules, a proportion $\alpha$ are agonists (so $0\leq \alpha \leq 1$) and the rest are antagonists. Antagonists remain attached for an exponentially-distributed duration with rate parameter $\mu_1$ (in units of per-seconds), while agonists remain attached for an exponentially-distributed duration with rate parameter $\mu_2$ (in units of per-seconds). An arriving molecule only becomes attached if the receptor is free of other molecules. 

In the long run, what is the percentage of time the receptor is occupied by an agonist? By an antagonist? Free?

\item \textbf{[BONUS]} For this part, assume there is only one type of ligand. The ligand binds to the receptor at rate $\lambda$ and unbinds at rate $\mu$. Once the ligand is bound, the receptor initiates an intracellular signal at rate $k$, but it can only do this while a ligand is bound. 

\begin{enumerate}
\item Write a 3-state continuous-time Markov transition matrix to describe this process. 

\item Suppose that at time $t=0$, there is a ligand bound to the receptor. How long on average until a signal is transduced?

\item Suppose that the receptor's kinetics are controlled by a control variable, so that $\lambda = c \lambda^\star$ and $\mu = c\mu^\star$. Again suppose the system starts with a ligand bound. Assume receptor kinetics have been slowed to approximately zero speed, so $c=0$, i.e., $\lambda=\mu=0$. In other words, the ligand is very unlikely to ever unbind.

\begin{enumerate}
\item According to your intuition, what is the mean time until a signal is transduced?
\item Set $\lambda=\mu=0$ in the transition matrix. What is the mean time until a signal is transduced? Does this agree with your intuition?
\item Set $\lambda=\mu=0$ in your answer from Part (b). What is the mean time until a signal is transduced? Does this agree with your intuition?
\end{enumerate}

\end{enumerate}


%%%%%%%%%%%%%%%%%%%%%%%%%%%%%%%%%%%%%% 
\end{enumerate}
\end{document}

%%%%%%%%%%%%%%%%%%%%%%%%%%%%%%%%%%%%%% 
